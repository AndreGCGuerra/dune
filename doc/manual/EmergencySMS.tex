% \newcommand{\spfdPFD}{\textbf{Position Fault Detection}}
% \newcommand{\spfdPET}{\textbf{Position Error Threshold}}
% \newcommand{\spfdPED}{\textbf{Position Error Delay}}
% \newcommand{\spfdPES}{\textbf{Position Error Samples}}
% \newcommand{\spfdETP}{\textbf{Error Throw Period}}

\section{SMS Emergency Mechanism}

\subsection{\secnameConfig}

Configuration parameters for this mechanism are located in \emph{...}.

\beginconfigtable
  \hline
  \tableheader
  \hline
  % \spfdPFD{} & PFD  & false & enables or disables position fault detections. \\
  % \hline
  % \spfdPET{} & PET & 0.2 & threshold of the error between the actuation command and servo actual position above which a detection is made. \\
  % \hline
  % \spfdPED{} & PED & 5.0 & amount of time during which the position error must remain above the threshold, so that a fault detection is triggered. \\
  % \hline
  % \spfdPES{} & PES & 5 & number of samples used in the moving average filtering the position error. \\
  % \hline
  % \spfdETP{} & ETP & 20.0 & cooldown time after which a current fault error can be thrown to the output \\
  \hline
\end{tabular}

\subsection{\secnameDescription}

...

\subsection{\secnameFlowchart}

Figure ...

% \begin{figure}[htbp]
% \begin{center}
% \includegraphics*[viewport= 147 380 409 670, scale=0.8]{figures/servopositionfault.pdf}
% \end{center}
% \caption{Servo position fault detection flowchart.}\label{fig:servoposfault}
% \end{figure}
