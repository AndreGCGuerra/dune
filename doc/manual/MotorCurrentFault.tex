\newcommand{\mcfdMCFD}{\textbf{Motor Current Fault Detection}}
\newcommand{\mcfdMCC}{\textbf{Motor Current Channel}}
\newcommand{\mcfdMCET}{\textbf{Motor Current Error Threshold}}
\newcommand{\mcfdMCD}{\textbf{Motor Current Delay}}
\newcommand{\mcfdMCAS}{\textbf{Motor Current Average Samples}}
\newcommand{\mcfdMCEP}{\textbf{Motor Current Error Period}}
\newcommand{\mcfdMCIP}{\textbf{Motor Current Interpolation Points}}

\section{Motor Current Fault Detection}

\subsection{\secnameConfig}

Configuration parameters for this mechanism are located in \emph{/etc/hardware/lmct/a500.ini}.

\beginconfigtable
  \hline
  \tableheader
  \hline
  \mcfdMCFD{} & MCFD & false & enables or disables position fault detections. \\
  \hline
  \mcfdMCC{} & MCC &  & ADC channel number used to measure the current. Default value depends on the MCTR version.\\
  \hline
  \mcfdMCET{} & MCET & 2.0 & Threshold for the error between the motor's current and the expected current. \\
  \hline
  \mcfdMCD{} & MCD & 5.0 & amount of time during which the error must remain above the threshold, so that a fault detection is triggered. \\
  \hline
  \mcfdMCAS{} & MCAS & 5 & number of samples used in the moving average filtering the position error. \\
  \hline
  \mcfdMCEP{} & MCEP & 20.0 & cooldown time after which a current fault error can be thrown to the output \\
  \hline
  \mcfdMCIP{} & MCIP &  & curve points used to create a model of the expected motor current. No default value. \\
  \hline
\end{tabular}

\subsection{\secnameDescription}

This mechanism uses a model of the motor current versus rpms in order to assess whether or not the motor is at a fault state. That model is a piecewise linear function put together using the \mcfdMCIP{}. The value of the function between each pair of points is computed using linear interpolation. \emph{The \mcfdMCIP{} are assumed to be in ascending order and in the format (rpm,current)}. Any number of points can be provided in the configuration file, as long as they are in an ascending order.

The motor current value and rpm measurements are constantly being read. The electric current estimated by the model for the rpms value being read, is compared against the electric current read by the ADCs. If the value being read is \mcfdMCET{} ampere above the expected value, and remains above it for more than \mcfdMCD{} seconds, a fault detection is triggered. If the last time that an error of this sort was thrown to the output happened more than \mcfdMCEP{} seconds ago, another error is thrown.

The diference being computed can be filtered with a moving average that will use \spfdPES{} samples for that purpose.

\subsection{\secnameFlowchart}

Figure~\ref{fig:motorcurrfault} depicts the flowchart of how the fault detection method works.
The variable \emph{error} represents the difference between the actual motor current being read and the current expected by the piecewise linear model referred above.

\begin{figure}[htbp]
\begin{center}
\includegraphics*[viewport= 147 380 409 670, scale=0.8]{figures/motorcurrentfault.pdf}
\end{center}
\caption{Motor current fault detection flowchart.}\label{fig:motorcurrfault}
\end{figure}
