\newcommand{\scfdCFD}{\textbf{Current Fault Detection~}}
\newcommand{\scfdCLT}{\textbf{Current Lower Threshold~}}
\newcommand{\scfdCUT}{\textbf{Current Upper Threshold~}}
\newcommand{\scfdMLF}{\textbf{Maximum Lower Faults~}}
\newcommand{\scfdMUF}{\textbf{Maximum Upper Faults~}}
\newcommand{\scfdFTC}{\textbf{Fault Time Cooldown~}}
\newcommand{\scfdETP}{\textbf{Error Throw Period~}}

\section{Servo Current Fault Detection}

\subsection{\secnameConfig}

Configuration parameters for this mechanism are located in \emph{/etc/hardware/lsct/a500.ini}.

\beginconfigtable
  \hline
  \tableheader
  \hline
  \scfdCFD & CFD  & false & enables or disables current fault detections. \\
  \hline
  \scfdCLT & CLT & 0.4 & lower threshold to trigger a minor current fault detection. \\
  \hline
  \scfdCUT & CUT & 0.7 & upper threshold to trigger a major current fault detection. \\
  \hline
  \scfdMLF & MLF & 20 & maximum number of minor faults that must be exceeded before a fault detection error is thrown \\
  \hline
  \scfdMUF & MUF & 4 & maximum number of major faults that must be exceeded before a fault detection error is thrown \\
  \hline
  \scfdFTC & FTC & 60.0 & cooldown time after which the counters of minor and major faults are reset \\
  \hline
  \scfdETP & ETP & 20.0 & cooldown time after which a current fault error can be thrown to the output \\
  \hline
\end{tabular}

\subsection{\secnameDescription}

This mecanism attempts to detect faults in the servos by reading the measurements of electric current flowing in them.
The current values are constantly being read for every servo. If the current exceeds the \scfdCLT, the minor faults counter is incremented. If the current exceeds the \scfdCUT, then both minor and major fault counters will be incremented. If a \scfdFTC amount of time passes before a fault detection is triggered, both minor and major fault counters are reset to zero.
If the values in the counters exceed, respectively, the \scfdMLF or \scfdMUF, there will be an attempt to throw an error to the output.
If the last time an error was thrown (for that same servo) was less than \scfdETP seconds ago, then the error will not be thrown. If more than that has passed then the error will go to the output. Note that this is valid for each servo independently. This means that if all servos enter a fault state and errors can be thrown, you will see in the output as many errors as the number of servos in the vehicle.

\subsection{\secnameFlowchart}

Figure~\ref{fig:servocurrfault} depicts the flowchart of how the fault detection method works. The example covers the faults related to the lower threshold, yet the detection for the upper threshold works in the exact same way. The letter ``I'' represents the value of the current being read for a certain servo.

\begin{figure}[htbp]
\begin{center}
\includegraphics*[viewport= 147 325 533 670, scale=1.0]{figures/servocurrentfault.pdf}
\end{center}
\caption{Servo current fault detection flowchart.}\label{fig:servocurrfault}
\end{figure}
