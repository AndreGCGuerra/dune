\documentclass[a4paper]{scrreprt}
\usepackage[utf8]{inputenc}

\usepackage[pdftex]{graphicx}

\usepackage[bookmarks=true,linkcolor=blue,colorlinks=true,citecolor=blue,urlcolor=blue]{hyperref}

\usepackage{amsmath}
\usepackage{bm}
\usepackage{amssymb}
\usepackage{amsthm}
\usepackage{enumerate}
\usepackage{colortbl}

\addtolength{\textwidth}{2in}
\addtolength{\hoffset}{-1in}
\addtolength{\textheight}{0in}
\addtolength{\voffset}{0in}

\renewcommand{\chaptername}{}

\begin{document}

\title{Fault Mechanisms Manual}

% author names and affiliations
% use a multiple column layout for up to three different
% affiliations
\author{LSTS}%

% make the title area
\maketitle

% Contents
\setcounter{tocdepth}{1}
\tableofcontents

\chapter{Introduction}
This document is intended to explain how the main emergency and fault mechanisms work in DUNE.

\chapter{Fault Mechanisms}

\footnotesize

\newcommand{\firstcolsize}{4cm}
\newcommand{\lastcolsize}{10cm}
\newcommand{\beginconfigtable}{\begin{tabular}{|p{\firstcolsize} | l | l | p{\lastcolsize}|}}
  \newcommand{\tableheader}{\rowcolor[gray]{0.8}{\bf{Parameter Name}} & \bf{Acronym} & \bf{Default} & \bf{Description} \\}
  \newcommand{\secnameConfig}{Configuration}
  \newcommand{\secnameDescription}{Description}
  \newcommand{\secnameFlowchart}{Flowchart}

\newcommand{\scfdCFD}{\textbf{Current Fault Detection~}}
\newcommand{\scfdCLT}{\textbf{Current Lower Threshold~}}
\newcommand{\scfdCUT}{\textbf{Current Upper Threshold~}}
\newcommand{\scfdMLF}{\textbf{Maximum Lower Faults~}}
\newcommand{\scfdMUF}{\textbf{Maximum Upper Faults~}}
\newcommand{\scfdFTC}{\textbf{Fault Time Cooldown~}}
\newcommand{\scfdETP}{\textbf{Error Throw Period~}}

\section{Servo Current Fault Detection}

\subsection{\secnameConfig}

Configuration parameters for this mechanism are located in \emph{/etc/hardware/lsct/a500.ini}.

\beginconfigtable
  \hline
  \tableheader
  \hline
  \scfdCFD & CFD  & false & enables or disables current fault detections. \\
  \hline
  \scfdCLT & CLT & 0.4 & lower threshold to trigger a minor current fault detection. \\
  \hline
  \scfdCUT & CUT & 0.7 & upper threshold to trigger a major current fault detection. \\
  \hline
  \scfdMLF & MLF & 20 & maximum number of minor faults that must be exceeded before a fault detection error is thrown \\
  \hline
  \scfdMUF & MUF & 4 & maximum number of major faults that must be exceeded before a fault detection error is thrown \\
  \hline
  \scfdFTC & FTC & 60.0 & cooldown time after which the counters of minor and major faults are reset \\
  \hline
  \scfdETP & ETP & 20.0 & cooldown time after which a current fault error can be thrown to the output \\
  \hline
\end{tabular}

\subsection{\secnameDescription}

This mecanism attempts to detect faults in the servos by reading the measurements of electric current flowing in them.
The current values are constantly being read for every servo. If the current exceeds the \scfdCLT, the minor faults counter is incremented. If the current exceeds the \scfdCUT, then both minor and major fault counters will be incremented. If a \scfdFTC amount of time passes before a fault detection is triggered, both minor and major fault counters are reset to zero.
If the values in the counters exceed, respectively, the \scfdMLF or \scfdMUF, there will be an attempt to throw an error to the output.
If the last time an error was thrown (for that same servo) was less than \scfdETP seconds ago, then the error will not be thrown. If more than that has passed then the error will go to the output. Note that this is valid for each servo independently. This means that if all servos enter a fault state and errors can be thrown, you will see in the output as many errors as the number of servos in the vehicle.

\subsection{\secnameFlowchart}

Figure~\ref{fig:servocurrfault} depicts the flowchart of how the fault detection method works. The example covers the faults related to the lower threshold, yet the detection for the upper threshold works in the exact same way. The letter ``I'' represents the value of the current being read for a certain servo.

\begin{figure}[htbp]
\begin{center}
\includegraphics*[viewport= 147 325 533 670, scale=1.0]{figures/servocurrentfault.pdf}
\end{center}
\caption{Servo current fault detection flowchart.}\label{fig:servocurrfault}
\end{figure}

\newcommand{\spfdPFD}{\textbf{Position Fault Detection}}
\newcommand{\spfdPET}{\textbf{Position Error Threshold}}
\newcommand{\spfdPED}{\textbf{Position Error Delay}}
\newcommand{\spfdPES}{\textbf{Position Error Samples}}
\newcommand{\spfdETP}{\textbf{Error Throw Period}}

\section{Servo Position Fault Detection}

\subsection{\secnameConfig}

Configuration parameters for this mechanism are located in \emph{/etc/hardware/lsct/a500.ini}.

\beginconfigtable
  \hline
  \tableheader
  \hline
  \spfdPFD{} & PFD  & false & enables or disables position fault detections. \\
  \hline
  \spfdPET{} & PET & 0.2 & threshold of the error between the actuation command and servo actual position above which a detection is made. \\
  \hline
  \spfdPED{} & PED & 5.0 & amount of time during which the position error must remain above the threshold, so that a fault detection is triggered. \\
  \hline
  \spfdPES{} & PES & 5 & number of samples used in the moving average filtering the position error. \\
  \hline
  \spfdETP{} & ETP & 20.0 & cooldown time after which a current fault error can be thrown to the output \\
  \hline
\end{tabular}

\subsection{\secnameDescription}

The position values of the servos (if available) are assumed to be synchronized with the command values, in other words, the adcs reading those positions are assumed to be properly calibrated.
Those positions are constantly being read. The error between them and the last command sent to the servo is computed.
The error value being read can be filtered with a moving average that will use \spfdPES{} samples for that purpose.
If that error goes above \spfdPET{} and remains above that value for more than \spfdPED{}, a position fault error will be thrown:
\begin{quote}
  potential fault in servo \#n, position error above X
\end{quote}
The same error will only be thrown to the output again if more than \spfdETP{} seconds have passed since the last error was thrown.


\subsection{\secnameFlowchart}

Figure~\ref{fig:servoposfault} depicts the flowchart of how the fault detection method works. The variable \emph{error} represents the absolute value of the diference between the servo's command and its actual position.

\begin{figure}[htbp]
\begin{center}
\includegraphics*[viewport= 147 380 409 670, scale=0.8]{figures/servopositionfault.pdf}
\end{center}
\caption{Servo position fault detection flowchart.}\label{fig:servoposfault}
\end{figure}

\newcommand{\mcfdMCFD}{\textbf{Motor Current Fault Detection}}
\newcommand{\mcfdMCC}{\textbf{Motor Current Channel}}
\newcommand{\mcfdMCET}{\textbf{Motor Current Error Threshold}}
\newcommand{\mcfdMCD}{\textbf{Motor Current Delay}}
\newcommand{\mcfdMCAS}{\textbf{Motor Current Average Samples}}
\newcommand{\mcfdMCEP}{\textbf{Motor Current Error Period}}
\newcommand{\mcfdMCIP}{\textbf{Motor Current Interpolation Points}}

\section{Motor Current Fault Detection}

\subsection{\secnameConfig}

Configuration parameters for this mechanism are located in \emph{/etc/hardware/lmct/a500.ini}.

\beginconfigtable
  \hline
  \tableheader
  \hline
  \mcfdMCFD{} & MCFD & false & enables or disables position fault detections. \\
  \hline
  \mcfdMCC{} & MCC &  & ADC channel number used to measure the current. Default value depends on the MCTR version.\\
  \hline
  \mcfdMCET{} & MCET & 2.0 & Threshold for the error between the motor's current and the expected current. \\
  \hline
  \mcfdMCD{} & MCD & 5.0 & amount of time during which the error must remain above the threshold, so that a fault detection is triggered. \\
  \hline
  \mcfdMCAS{} & MCAS & 5 & number of samples used in the moving average filtering the position error. \\
  \hline
  \mcfdMCEP{} & MCEP & 20.0 & cooldown time after which a current fault error can be thrown to the output \\
  \hline
  \mcfdMCIP{} & MCIP &  & curve points used to create a model of the expected motor current. No default value. \\
  \hline
\end{tabular}

\subsection{\secnameDescription}

This mechanism uses a model of the motor current versus rpms in order to assess whether or not the motor is at a fault state. That model is a piecewise linear function put together using the \mcfdMCIP{}. The value of the function between each pair of points is computed using linear interpolation. \emph{The \mcfdMCIP{} are assumed to be in ascending order and in the format (rpm,current)}. Any number of points can be provided in the configuration file, as long as they are in an ascending order.

The motor current value and rpm measurements are constantly being read. The electric current estimated by the model for the rpms value being read, is compared against the electric current read by the ADCs. If the value being read is \mcfdMCET{} ampere above the expected value, and remains above it for more than \mcfdMCD{} seconds, a fault detection is triggered. If the last time that an error of this sort was thrown to the output happened more than \mcfdMCEP{} seconds ago, another error is thrown.

The diference being computed can be filtered with a moving average that will use \spfdPES{} samples for that purpose.


\subsection{\secnameFlowchart}

Figure~\ref{fig:motorcurrfault} depicts the flowchart of how the fault detection method works.
The variable \emph{error} represents the difference between the actual motor current being read and the current expected by the piecewise linear model referred above.

\begin{figure}[htbp]
\begin{center}
\includegraphics*[viewport= 147 380 409 670, scale=0.8]{figures/motorcurrentfault.pdf}
\end{center}
\caption{Motor current fault detection flowchart.}\label{fig:motorcurrfault}
\end{figure}

% \newcommand{\spfdPFD}{\textbf{Position Fault Detection}}
% \newcommand{\spfdPET}{\textbf{Position Error Threshold}}
% \newcommand{\spfdPED}{\textbf{Position Error Delay}}
% \newcommand{\spfdPES}{\textbf{Position Error Samples}}
% \newcommand{\spfdETP}{\textbf{Error Throw Period}}

\section{SMS Emergency Mechanism}

\subsection{\secnameConfig}

Configuration parameters for this mechanism are located in \emph{...}.

\beginconfigtable
  \hline
  \tableheader
  \hline
  % \spfdPFD{} & PFD  & false & enables or disables position fault detections. \\
  % \hline
  % \spfdPET{} & PET & 0.2 & threshold of the error between the actuation command and servo actual position above which a detection is made. \\
  % \hline
  % \spfdPED{} & PED & 5.0 & amount of time during which the position error must remain above the threshold, so that a fault detection is triggered. \\
  % \hline
  % \spfdPES{} & PES & 5 & number of samples used in the moving average filtering the position error. \\
  % \hline
  % \spfdETP{} & ETP & 20.0 & cooldown time after which a current fault error can be thrown to the output \\
  \hline
\end{tabular}

\subsection{\secnameDescription}

...


\subsection{\secnameFlowchart}

Figure ...

% \begin{figure}[htbp]
% \begin{center}
% \includegraphics*[viewport= 147 380 409 670, scale=0.8]{figures/servopositionfault.pdf}
% \end{center}
% \caption{Servo position fault detection flowchart.}\label{fig:servoposfault}
% \end{figure}

% \bibliographystyle{plain}
% \bibliography{refs}

\end{document}
